%% LyX trick_preamble_code_into_believing_that_this_was_created_by_lyx created this file.  For more info, see http://www.lyx.org/.
%% Do not edit unless you really know what you are doing.
\documentclass[english]{book}
\usepackage[T1]{fontenc}
\usepackage[latin9]{inputenc}
\usepackage{refstyle}
\usepackage{float}
\usepackage{amsthm}

\makeatletter

%%%%%%%%%%%%%%%%%%%%%%%%%%%%%% LyX specific LaTeX commands.

\AtBeginDocument{\providecommand\partref[1]{\ref{part:#1}}}
\AtBeginDocument{\providecommand\chapref[1]{\ref{chap:#1}}}
\AtBeginDocument{\providecommand\secref[1]{\ref{sec:#1}}}
\AtBeginDocument{\providecommand\subref[1]{\ref{sub:#1}}}
\AtBeginDocument{\providecommand\parref[1]{\ref{par:#1}}}
\AtBeginDocument{\providecommand\figref[1]{\ref{fig:#1}}}
\AtBeginDocument{\providecommand\tabref[1]{\ref{tab:#1}}}
\AtBeginDocument{\providecommand\algref[1]{\ref{alg:#1}}}
\AtBeginDocument{\providecommand\fnref[1]{\ref{fn:#1}}}
\AtBeginDocument{\providecommand\enuref[1]{\ref{enu:#1}}}
\AtBeginDocument{\providecommand\eqref[1]{\ref{eq:#1}}}
\AtBeginDocument{\providecommand\lemref[1]{\ref{lem:#1}}}
\AtBeginDocument{\providecommand\thmref[1]{\ref{thm:#1}}}
\AtBeginDocument{\providecommand\corref[1]{\ref{cor:#1}}}
\AtBeginDocument{\providecommand\propref[1]{\ref{prop:#1}}}
\floatstyle{ruled}
\newfloat{algorithm}{tbp}{loa}[chapter]
\providecommand{\algorithmname}{Algorithm}
\floatname{algorithm}{\protect\algorithmname}
\RS@ifundefined{subref}
  {\def\RSsubtxt{section~}\newref{sub}{name = \RSsubtxt}}
  {}
\RS@ifundefined{thmref}
  {\def\RSthmtxt{theorem~}\newref{thm}{name = \RSthmtxt}}
  {}
\RS@ifundefined{lemref}
  {\def\RSlemtxt{lemma~}\newref{lem}{name = \RSlemtxt}}
  {}


%%%%%%%%%%%%%%%%%%%%%%%%%%%%%% Textclass specific LaTeX commands.
\theoremstyle{plain}
\newtheorem{thm}{\protect\theoremname}
\theoremstyle{plain}
\newtheorem{lem}[thm]{\protect\lemmaname}
\theoremstyle{plain}
\newtheorem{cor}[thm]{\protect\corollaryname}
\theoremstyle{plain}
\newtheorem{prop}[thm]{\protect\propositionname}
\theoremstyle{plain}
\newtheorem{conjecture}[thm]{\protect\conjecturename}
\theoremstyle{plain}
\newtheorem{fact}[thm]{\protect\factname}
\theoremstyle{definition}
\newtheorem{defn}[thm]{\protect\definitionname}
\theoremstyle{definition}
\newtheorem{example}[thm]{\protect\examplename}
\theoremstyle{definition}
\newtheorem{problem}[thm]{\protect\problemname}
\theoremstyle{definition}
\newtheorem{xca}[thm]{\protect\exercisename}
\theoremstyle{remark}
\newtheorem{rem}[thm]{\protect\remarkname}
\theoremstyle{remark}
\newtheorem{claim}[thm]{\protect\claimname}
\ifx\proof\undefined
\newenvironment{proof}[1][\protect\proofname]{\par
\normalfont\topsep6\p@\@plus6\p@\relax
\trivlist
\itemindent\parindent
\item[\hskip\labelsep\scshape #1]\ignorespaces
}{%
\endtrivlist\@endpefalse
}
\providecommand{\proofname}{Proof}
\fi

\makeatother

\usepackage{babel}
\providecommand{\claimname}{Claim}
\providecommand{\conjecturename}{Conjecture}
\providecommand{\corollaryname}{Corollary}
\providecommand{\definitionname}{Definition}
\providecommand{\examplename}{Example}
\providecommand{\exercisename}{Exercise}
\providecommand{\factname}{Fact}
\providecommand{\lemmaname}{Lemma}
\providecommand{\problemname}{Problem}
\providecommand{\propositionname}{Proposition}
\providecommand{\remarkname}{Remark}
\providecommand{\theoremname}{Theorem}

\begin{document}

\part{part\label{part:part}}

\partref{part}


\chapter{chapter\label{chap:chapter}}

\chapref{chapter}


\section{Section\label{sec:Section}}

\secref{Section}


\subsection{Subsection\label{sub:subsection}}

Ref to \subref{subsection}.


\subsubsection{Subsubsection\label{sub:Subsubsection}}

Ref to \subref{Subsubsection}.


\paragraph{paragraph\label{par:paragraph}}

\parref{paragraph}


\subparagraph{subparagraph\label{par:subparagraph}}

\parref{subparagraph}

\begin{figure}
\protect\caption{figure\label{fig:figure}}
\end{figure}


\figref{figure}

\begin{table}
\protect\caption{table\label{tab:table}}
\end{table}


\tabref{table}

\begin{algorithm}
\protect\caption{algorithm\label{alg:algorithm}}
\end{algorithm}


\algref{algorithm}

\footnote{foot\label{fn:foot}%
}

\fnref{foot}

\marginpar{margin\label{margin}}

\ref{margin}
\begin{enumerate}
\item item\label{enu:item}
\end{enumerate}
\enuref{item}

\begin{equation}
A=B\label{eq:A=00003DB}
\end{equation}


\eqref{A=00003DB}
\begin{lem}
lemma\label{lem:lemma}

\lemref{lemma}\end{lem}
\begin{thm}
theorem\label{thm:theorem}

\thmref{theorem}\end{thm}
\begin{cor}
corollary\label{cor:corollary}

\corref{corollary}\end{cor}
\begin{prop}
proposition\label{prop:proposition}

\propref{proposition}\end{prop}
\begin{conjecture}
conjecture\label{conjecture}

\ref{conjecture}\end{conjecture}
\begin{fact}
fact\label{fact}

\ref{fact}\end{fact}
\begin{defn}
definition\label{definition}

\ref{definition}\end{defn}
\begin{example}
example\label{example}

\ref{example}\end{example}
\begin{problem}
problem\label{problem}

\ref{problem}\end{problem}
\begin{xca}
exercise\label{exercise}

\ref{exercise}\end{xca}
\begin{rem}
remark\label{remark}

\ref{remark}\end{rem}
\begin{claim}
claim\label{claim}

\ref{claim}\end{claim}
\begin{proof}
proof\label{proof}

\ref{proof}\end{proof}

\end{document}
