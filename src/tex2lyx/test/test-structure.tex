%% LyX trick_preamble_code_into_believing_that_this_was_created_by_lyx created this file.  For more info, see http://www.lyx.org/.
%% Do not edit unless you really know what you are doing.
\documentclass[legalpaper,francais,german,newzealand]{article}

%%%%%%%%%%%%%%%%%%%%%%%%%%%%%% LyX specific LaTeX commands.

\usepackage{babel}
\newcommand{\noun}[1]{\textsc{#1}}

%%%%%%%%%%%%%%%%%%%%%%%%%%%%%% Textclass specific LaTeX commands.

\newenvironment{lyxlist}[1]
	{\begin{list}{}
		{\settowidth{\labelwidth}{#1}
		\setlength{\leftmargin}{\labelwidth}
		\addtolength{\leftmargin}{\labelsep}
		\renewcommand{\makelabel}[1]{##1\hfil}}}
	{\end{list}}

% LyX will also recognize this command:
% \@ifundefined{date}{}{\date{}}
% and also this:
% \date{}
%%%%%%%%%%%%%%%%%%%%%%%%%%%%%% User specified LaTeX commands.

\newenvironment{foo}{==[}{]==}

\begin{document}

\title{Title}

\date{}

\maketitle
This document contains all sorts of layouts we are supposed to
support, along with weird nestings.

At time you will see that I use subsubsections in weird places. The
intent is just to make sure that I can include a macro-type layout
everyzhere it makes sense.

A normal paragraph
\begin{equation}
x = \sin y
\end{equation}
with maths inside it.

\begin{quote}
An environment...

... with two paragraphs
\end{quote}

\begin{foo}
an unknown environment
\end{foo}


\section{A section}

\section[Hello!]{A section with optional argument}

\begin{quote}
An environment
\end{quote}

\section*{A starred section}

\begin{figure}
\caption{ \emph{\noun{is}} a caption}
\end{figure}

\begin{figure}
\caption[s\noun{ho}rt]{this \emph{is} a caption}
\end{figure}

A paragraph\footnote{hello} with a footnote and another
one\footnote{hello

there} with several paragraphs

some ERT \vspace{1cm} aa

and another paragraph

\begin{center}
Some centered stuff (does not work)
\end{center}

\begin{quotation}
An environment

\subsubsection*{with a command inside it}
\end{quotation}

\begin{quotation}
Another environment

\begin{quotation}
With another one inside it (with same layout)

[this one even has several paragraphs!]
\end{quotation}

\end{quotation}

We can also nest enumerations

\begin{enumerate}
\item Item1
\begin{enumerate}
\item Item1.a

\item Item1.b (there is  a paragraph break in front of this)
\begin{itemize}
\item Item1.b.*
\item Item1.b.*
\end{itemize}
\end{enumerate}
\item Item2
\end{enumerate}
\begin{enumerate}
\item Item1 (appears as Item3 with bug 5716)

Normal paragraph in Item1

\begin{enumerate}
\item Item1.a
\end{enumerate}
\end{enumerate}

Let's see what happens when normal paragraphs are inserted in lists:

\begin{itemize}
\item the first item

with some explanatory text under it

and a second paragraph for good measure

\subsubsection*{we can even have one as a subsubsection}

\item the second item

\item the third item

\subsubsection*{and a sssection heading inside it (why not?)}
\end{itemize}

What else? Well, we have descriptions:
\begin{description}
\item[ABC] first item
\item[BCD] second one
\end{description}
labelings:
\begin{lyxlist}{00.00.0000}
\item [label~1] first item
\item [label~2] second item
\end{lyxlist}
and bibliography:
\begin{thebibliography}{9}
\bibitem{FOO} Edward Bar. \emph{The Foo Book}. (1999)
\bibitem{FO2} Walter Müller \emph{The Müller Book}. (2004) 
\end{thebibliography}

\appendix

\section{This is the Appendix}

\noindent blabla bla bla

switch to german:\selectlanguage{german}
Hallo!
\foreignlanguage{french}{some \emph{french}}
back to english:\selectlanguage{newzealand}
and some nested \foreignlanguage{francais}{french \foreignlanguage{german}{nested
\emph{german}} french} english stuff.
Note that we both used \texttt{french} and the \texttt{francais} alias for the
french text, but for some reason this does not work with the
\texttt{newzealand} alias and \texttt{english} for english text.

\section{Another Appendix section}

blub

Test for missing \textbackslash end\_deeper (file format 278).
This must stay at the very end of the document!
\begin{itemize}
\item par1

par2
\begin{enumerate}
\item par1

par2
\end{enumerate}

\end{itemize}

\end{document}
