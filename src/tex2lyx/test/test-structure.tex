%% LyX trick_preamble_code_into_believing_that_this_was_created_by_lyx created this file.  For more info, see http://www.lyx.org/.
%% Do not edit unless you really know what you are doing.
\documentclass[legalpaper,french,german,english,dummyoption]{article}
\usepackage[latin9]{inputenc}
\usepackage{color}
\usepackage{rotfloat}
\usepackage{wrapfig}

\makeatletter

%%%%%%%%%%%%%%%%%%%%%%%%%%%%%% LyX specific LaTeX commands.
\pdfpageheight\paperheight
\pdfpagewidth\paperwidth

\providecommand{\LyX}{L\kern-.1667em\lower.25em\hbox{Y}\kern-.125emX\@}
\newcommand{\noun}[1]{\textsc{#1}}
%% Because html converters don't know tabularnewline
\providecommand{\tabularnewline}{\\}
\newcommand{\lyxadded}[3]{#3}
\newcommand{\lyxdeleted}[3]{}

%%%%%%%%%%%%%%%%%%%%%%%%%%%%%% Textclass specific LaTeX commands.
\newenvironment{lyxlist}[1]
{\begin{list}{}
{\settowidth{\labelwidth}{#1}
 \setlength{\leftmargin}{\labelwidth}
 \addtolength{\leftmargin}{\labelsep}
 \renewcommand{\makelabel}[1]{##1\hfil}}}
{\end{list}}
\newcommand{\strong}[1]{\textbf{#1}}

%%%%%%%%%%%%%%%%%%%%%%%%%%%%%% User specified LaTeX commands.
\newenvironment{foo}{==[}{]==}

\usepackage{fixltx2e} % this should cause the fixltx2e module to be loaded

\date{}

\@ifundefined{showcaptionsetup}{}{%
 \PassOptionsToPackage{caption=false}{subfig}}
\usepackage{subfig}
\makeatother

\usepackage{babel}
\addto\extrasfrench{%
   \providecommand{\og}{\leavevmode\flqq~}%
   \providecommand{\fg}{\ifdim\lastskip>\z@\unskip\fi~\frqq}%
}

\begin{document}

\title{Title}

% this should be recognized as empty date:
\date{}

\maketitle
This document contains all sorts of layouts we are supposed to
support, along with weird nestings.

At time you will see that I use subsubsections in weird places. The
intent is just to make sure that I can include a macro-type layout
everyzhere it makes sense.

A normal paragraph
\begin{equation}
x = \sin y
\end{equation}
with maths inside it.

\begin{quote}
An environment...

... with two paragraphs
\end{quote}

\begin{foo}
an unknown environment
\end{foo}


\section{A section}

\section[Hello!]{A section with optional argument}

This causes the \strong{logikalmkup} module to be loaded.

\begin{quote}
An environment
\end{quote}

We also support change tracking:
\lyxadded{Hans Wurst}{Sun Nov  6 10:39:39 2011}{Added text}
some parts remain
\lyxdeleted{Hans Wurst}{Sun Nov  6 10:39:55 2011}{This was the original text}
some parts remain

\section*{A starred section for floats}

\begin{figure}
\caption{ \emph{\noun{is}} a caption}
\end{figure}

\begin{figure}
\caption[s\noun{ho}rt]{this \emph{is} a caption}
\end{figure}

\begin{sidewaystable*}
\caption{rotated table, spanning all columns}

\centering{}%
\begin{tabular}[b]{|c|c|}
\hline 
a  & b\tabularnewline
\hline 
\hline 
d  & c\tabularnewline
\hline 
\end{tabular}
\end{sidewaystable*}

\begin{wrapfigure}[4]{L}[2ex]{0.5\columnwidth}%
\begin{centering}
fdgsdfdh
\par\end{centering}

\caption{test1}
\end{wrapfigure}%
\LyX{} is a document preparation system. It excels at letting you
create complex technical and scientific articles with mathematics,
cross-references, bibliographies, indices, etc. It is very good at
documents of any length in which the usual processing abilities are
required: automatic sectioning and pagination, spell checking, and
so forth. It can also be used to write a letter to your mom, though
granted, there are probably simpler programs available for that. It
is definitely not the best tool for creating banners, flyers, or advertisements
(we'll explain why later), though with some effort all these can be
done, too.

\begin{wrapfigure}{o}{0.5\columnwidth}%
\begin{centering}
fdgs
\par\end{centering}

\caption{test2}
\end{wrapfigure}%
\LyX{} is a document preparation system. It excels at letting you
create complex technical and scientific articles with mathematics,
cross-references, bibliographies, indices, etc. It is very good at
documents of any length in which the usual processing abilities are
required: automatic sectioning and pagination, spell checking, and
so forth. It can also be used to write a letter to your mom, though
granted, there are probably simpler programs available for that. It
is definitely not the best tool for creating banners, flyers, or advertisements
(we'll explain why later), though with some effort all these can be
done, too.

\begin{wraptable}{i}[0.05\textwidth]{5ex}%
\caption{fdg}


\centering{}dfgd\end{wraptable}%
\LyX{} is a document preparation system. It excels at letting you
create complex technical and scientific articles with mathematics,
cross-references, bibliographies, indices, etc. It is very good at
documents of any length in which the usual processing abilities are
required: automatic sectioning and pagination, spell checking, and
so forth. It can also be used to write a letter to your mom, though
granted, there are probably simpler programs available for that. It
is definitely not the best tool for creating banners, flyers, or advertisements
(we'll explain why later), though with some effort all these can be
done, too.

\begin{table}
\caption{lk�l��}


\subfloat[�lk�l�]{

kl��}

\end{table}


\begin{figure}
\subfloat[te\%st]{sub\textcolor{red}{fig}u\%re 1

}

\hfill{} \subfloat[]{subfigure 2�

}

\subfloat{subfigure 3}

\caption{figure caption}


strange usage, but valid 
\end{figure}


\subsection{Some paragraph stuff}

A paragraph\footnote{hello} with a footnote and another
one\footnote{hello

there} with several paragraphs \vspace{1cm} aa

and another paragraph

\begin{center}
Some centered stuff
\end{center}

\begin{quotation}
An environment

\subsubsection*{with a command inside it}
\end{quotation}

\begin{quotation}
Another environment

\begin{quotation}
With another one inside it (with same layout)

[this one even has several paragraphs!]
\end{quotation}

\end{quotation}

We can also nest enumerations

\begin{enumerate}
\item Item1
\begin{enumerate}
\item Item1.a

\item Item1.b (there is  a paragraph break in front of this)
\begin{itemize}
\item Item1.b.*
\item Item1.b.*
\end{itemize}
\end{enumerate}
\item Item2
\end{enumerate}
\begin{enumerate}
\item Item1 (appears as Item3 with bug 5716)

Normal paragraph in Item1

\begin{enumerate}
\item Item1.a
\end{enumerate}
\end{enumerate}

Let's see what happens when normal paragraphs are inserted in lists:

\begin{itemize}
\item the first item

with some explanatory text under it

and a second paragraph for good measure

\subsubsection*{we can even have one as a subsubsection}

\item the second item

\item the third item

\subsubsection*{and a sssection heading inside it (why not?)}
\end{itemize}

What else? Well, we have descriptions:
\begin{description}
\item[ABC] first item
\item[BCD] second one
\item[{x y z}] with space
\item  % hi there
[{x y % bla
z}] and with comments
\end{description}
labelings:
\begin{lyxlist}{00.00.0000}
\item [label~1] first item
\item [label~2] second item
\end{lyxlist}
and bibliography:
\begin{thebibliography}{9}
\bibitem{FOO} Edward Bar. \emph{The Foo Book}. (1999)
\bibitem{FO2} Walter M�ller \emph{The M�ller Book}. (2004) 
\end{thebibliography}

\appendix

\section{This is the Appendix}

\noindent blabla bla bla

switch to german:\selectlanguage{german}
Hallo!
\foreignlanguage{french}{some \emph{french}}
back to english:\selectlanguage{newzealand}
and some nested \foreignlanguage{francais}{french \foreignlanguage{german}{nested
\emph{german}} french} english stuff.
Note that we both used \texttt{french} and the \texttt{francais} alias for the
french text, but for some reason this does not work with the
\texttt{newzealand} alias and \texttt{english} for english text.

\section{Another Appendix section}

blub

Test for missing \textbackslash end\_deeper (file format 278).
This must stay at the very end of the document!
\begin{itemize}
\item par1

par2
\begin{enumerate}
\item par1

par2
\end{enumerate}

\end{itemize}

\end{document}
